\documentclass[onecolumn]{article}
\usepackage[latin1]{inputenc}
\usepackage{enumerate}
\usepackage{hyperref}
\usepackage{graphics}
\usepackage{graphicx}
\usepackage{caption}
\usepackage{subcaption}
\usepackage{tabularx}
\usepackage{amsmath}
\usepackage{multicol}
% enumerate is numbered \begin{enumerate}[(I)] is cap roman in parens
% itemize is bulleted \begin{itemize}
% subfigures:
% \begin{subfigure}[b]{0.5\textwidth} \includegraphics{asdf.jpg} \caption{} \label{subfig:asdf} \end{subfigure}
\hypersetup{colorlinks=true, urlcolor=blue, linkcolor=blue, citecolor=red}
\newcommand{\awesome}[1]{{\Huge #1 is \textbf{awesome!}}}
\graphicspath{ {C:/Users/Evan/Desktop/} }
\title{Assignment 2: Week 3}
\author{Evan Ott}
\date{\today}
\setcounter{secnumdepth}{1}

\usepackage[parfill]{parskip}
\begin{document}
\maketitle
\begin{abstract}
The purpose of this document is to test your skills for the third week of the SPS \LaTeX seminar. It is much more minimal than last week because most of the structures referenced in the learning
module can be added into existing documents.
\end{abstract}
\tableofcontents
\begin{multicols}{2}
\begin{section}{Tables and Matrices}
As stated in the presentation, please refer to the previous week's document available at \url{www.ph.utexas.edu/~sps/LaTeX/} for practice using tables and matrices.
\end{section}
\end{multicols}

\begin{section}{References and Citations}
Your first task is to use some labels. An ideal way to do this is to use your assignment from last week, add a label, then reference it from text somewhere else. To do this, you may find \cite{sps} helpful.
\end{section}

\begin{section}{Formatting}
\begin{subsection}{Equations}
	You should try recreating the Binomial Approximation (Equation \ref{eq:alignedeq}) in a document of your own:
	\begin{align}
		f(x)&=(1+x)^n \notag \\
		&=\sum_{i=0}^{n}\binom{n}{i}x^i  \notag \\
		&=1+nx+\frac{n(n+1)}{2}x^2+\cdots \notag \\
		&\approx1+nx\label{eq:alignedeq}
	\end{align}
	(HINT: You may find \verb+\binom+, \verb+\cdots+ and \verb+\approx+ useful.)
\end{subsection}

\begin{subsection}{Images}
	While subfigures may be a bit more esoteric, you should try to scale some images or at least try putting a caption on the image you used last time. The cool thing about subfigures
	though is that you can refer to them separately like in Figure \ref{fig:fdsaa}. There does appear to be an issue that the \verb+\label{...}+ command must be the VERY LAST part
	of the thing you're referencing for it to work - so you'd have the label, then the \verb+\end{enclosing_environment}+ following directly behind it.
	\begin{figure}
		\begin{subfigure}[b]{.5\textwidth}
			\centering
			\includegraphics[width=.5in, height=2in]{sps_logo.png}
			\caption{The SPS logo, now in VERTICAL}\label{fig:fdsaa}
		\end{subfigure}
		\begin{subfigure}[b]{.5\textwidth}
			\centering
			\includegraphics[width=2in, height=2in]{sps_logo.png}
			\caption{The SPS logo, now back to squares...}
		\end{subfigure}
	\caption{This is a figure composed of subfigures! Check out Figure \ref{fig:fdsaa}}\label{fig:fdsa}
	\end{figure}
\end{subsection}

\begin{subsection}{Sections}
	You should play with making subsections where need be, and with the counter for numbering sections. This should be reflected in the table of contents if you choose to make one.
	For example, this document only numbers \verb+section+ environments.
\end{subsection}

\begin{subsection}{URLs}
	You should try including some urls, like to \url{google.com} or to the \href{www.ph.utexas.edu/~sps/}{SPS Homepage}.
\end{subsection}

\begin{subsection}{Citations}
	You should be able to reproduce the citation listed for Einstein's Brownian Motion paper as seen in the References and cite it~\cite{einstein}.
\end{subsection}
\begin{subsection}{Lists}
	\begin{enumerate}[A.]
		\item{You should create some lists.}
		\begin{itemize}
			\item{It will help you with code structuring}
			\item{It will make sure you know how to do things in \LaTeX.}
			\end{itemize}
		\item{You should then do other things}
		\item{You should note that a list within a list should not be contained in an \verb+\item+.}
		\item{You should then go watch \href{http://www.youtube.com/watch?v=NfPndEB2ec0}{this video}.}
	\end{enumerate}
\end{subsection}
\begin{subsection}{Multicolumn}
	Try making your document multi-column in some parts. It's kinda funky if you don't have a lot of text in the part you want multi-columned. So, you should probably use some lorem ipsum
	text. You can find an space-themed lorem ipsum at \url{http://spaceipsum.com/}.
\end{subsection}
\end{section}
\begin{section}{Programming}
	This section is for people who really want to be able to do cool things in \LaTeX. It's not needed for most papers, but can make life easier sometimes. As such, your assignment is small.
	Define a command so that you can write \verb+\awesome{Evan Ott}+ which will print \awesome{Evan Ott}. Obviously, you  could put in any name and it would print that name instead.
	You may find the \verb+\Huge+ command helpful. You should also make sure the text following your \verb+\awesome{name}+ command doesn't come out huge.
\end{section}
\begin{section}{Conclusions}
	At this point, you should be well versed in the ways of \LaTeX. After this, it's mostly solving weird issues that come up when you try to outsmart \LaTeX. But that's okay. I have to
	use symbol sheets all the time, and used various Google searches all throughout crafting this presentation and document. As long as you can do it, and can figure out why it works,
	you're pretty much set. As always, you can send questions to the SPS officers if you have any: \url{mailto:spsofficers@gmail.com} will open in your native mail client, or you can copy
	the plain text spsofficers@gmail.com
\end{section}
\begin{thebibliography}{9}
\bibitem{sps}
Society of Physics Students UT Austin Chapter. \emph{LaTeX Week Three Presentation}. (2013). \url{www.ph.utexas.edu/~sps/LaTeX/Week3.pdf}
\bibitem{einstein}
A.~Einstein; trans. A.~D.~Cowper \textbf{On the Movement of Small Particles Suspended in a Stationary Liquid Demanded by the Molecular-Kinetic Theory of Heat}.
	\emph{Investigations on the theory of the Brownian motion}. Dover Publications. (1926). \url{http://users.physik.fu-berlin.de/~kleinert/files/eins_brownian.pdf}
\end{thebibliography}
\end{document}