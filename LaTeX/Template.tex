\documentclass{article}
\usepackage[latin1]{inputenc}
\usepackage{enumerate}
\usepackage{hyperref}
\usepackage{graphics}
\usepackage{graphicx}
\usepackage{caption}
\usepackage{subcaption}
% enumerate is numbered \begin{enumerate}[(I)] is cap roman in parens
% itemize is bulleted \begin{itemize}
% subfigures:
% \begin{subfigure}[b]{0.5\textwidth} \includegraphics{asdf.jpg} \caption{} \label{subfig:asdf} \end{subfigure}
\hypersetup{colorlinks=true, urlcolor=blue, linkcolor=blue, citecolor=red}
\graphicspath{ {C:/Users/YOUR_USER_NAME/FOLDERS_HERE/} }
\title{TITLE}
\author{YOUR NAME \\ OTHER INFO}
\date{\today}
\setcounter{secnumdepth}{0}	% gets rid of "x.y" in front of sections and subsections.
\usepackage[parfill]{parskip}
\begin{document}
\maketitle

%%%%%%%%%%%%%%%%%%%%%%%%%%%%%%%%%%%%%%%%%%%
%%%%%%%%%%%%%%%%%%%%%%%%%%%%%%%%%%%%%%%%%%%
%
% If you want to keep this as your template, but not have everything print, you can just comment
% out all the lines you don't want with a % sign in front of the part of the line you don't want.
% In TeXworks, you can select multiple lines and hit Ctrl+Shift+] or Ctrl+Shift+[ to comment or
% uncomment lines. This is only a crummy template. You will want to do things your own way.
%
%%%%%%%%%%%%%%%%%%%%%%%%%%%%%%%%%%%%%%%%%%%
%%%%%%%%%%%%%%%%%%%%%%%%%%%%%%%%%%%%%%%%%%%

\begin{section}{SECTIONNAME}
	asdf
	\begin{subsection}{SUBSECTIONNAME}
		fdsa
		\begin{subsubsection}{SUBSUBSECTIONNAME}
			asdf
		\end{subsubsection}
	\end{subsection}
\end{section}

\begin{section}{EXAMPLES}
\begin{subsection}{equations}
\begin{equation}
	\label{eq:equationTemplate}
	a=a
\end{equation}

As we saw in Equation \ref{eq:equationTemplate}, math is always easy*. We can also do math this way: $a=b$.
\end{subsection}

%\begin{subsection}{figures}
%\begin{figure}
% \label{fig:figureTemplate}
% \includegraphics{graphics_name.ext}
% \caption{my caption}
%\end{figure}

% As we would see in Figure \ref{fig:figureTemplate}, if we had a picture, we could reference it.
%\end{subsection}

\begin{subsection}{tables}
	\begin{tabular}{ l | c | r}
		\multicolumn{3}{c}{Centered heading over three columns} \\
		\hline %horizontal line
		data$_1$ & data$_2$ & data$_3$ \\
		1 & 2 & 3
	\end{tabular}
\end{subsection}

\begin{subsection}{matrices}
	\begin{equation}
		\left(\begin{array}{ccc}
		1 & 2& 3 \\
		4 & 5 &6 \end{array}\right)
	\end{equation}
	That matrix uses the ( and ) as the left and right endcaps for the matrix. You can use other symbols like [ and ] or $\left|,\right|$. Experiment.
\end{subsection}

\begin{subsection}{end?}
	That's all for now. Check back later for more quick examples in a pseudo-templated form.
\end{subsection}

\end{section}
\end{document}