%=====================================================
%====== If you are new to LaTeX, this website ========
%======     will be your new best friend:     ========
%======   http://en.wikibooks.org/wiki/LaTeX  ========
%======   Template created by Jonathan Blair  ========
%=====================================================



%=====================================================
%============ Controls ===============================
%=====================================================

%\documentclass[12pt,letterpaper,onecolumn]{article}
\documentclass[11pt,letterpaper,onecolumn]{article}
%\documentclass[10pt,letterpaper,onecolumn]{article}  % not recommended
%\documentclass[12pt,letterpaper,twocolumn]{article}
%\documentclass[11pt,letterpaper,twocolumn]{article}
%\documentclass[10pt,letterpaper,twocolumn]{article}


%\usepackage{amsmath}
\usepackage{graphics}
%\graphicspath{{path-to-folder-containing-necessary-graphics}{other folder as necessary}}


%=====================================================
%============ \begin{document} =======================
%=====================================================

\begin{document}

%=====================================================
%============ Title ==================================
%=====================================================

\title{\bf Observation of Very Interesting Phenomena}
%\title{\Large\bf Larger, Bolded Title}

%=====================================================
%============ Author =================================
%=====================================================
\author{
 Very Good Student \\*
  \\*
 PHY 353L Modern Laboratory \\*
 Department of Physics \\*
 The University of Texas at Austin \\*
 Austin, TX 78712, USA
}
\date{February 30, 2012}

%\address{The University of Texas, Austin, Texas, 78712}

\maketitle

%=====================================================
%============ Abstract ===============================
%=====================================================

\begin{abstract}

In an experiment designed to study very interesting phenomena
we have observed fascinating events. Our measurements, conducted
with the PHY353L instruments, yield values of the velocity of neutrinos 
equal to $(v_\nu - c)/c  = (0.05\pm0.12)$. Our results are consistent with earlier 
observations and confirm our understanding of the underlying physics.

\end{abstract}

%=====================================================
%============ Body of the article ==========================
%=====================================================

%=====================================================
%============ Section ==================================
%=====================================================

\section{Introduction} 

\subsection{Physics Motivation}

{\it (A paragraph)} Broad physics motivation should be discussed briefly but
meaningfully. Basic phenomena should be
explained (or referred to) and 
prediction for experimental results clearly
stated. Here and throughout the report appropriate
references should be included~\cite{book, article}.

\subsection{Theoretical background}

{\it (A paragraph)}  Provide some more theoretical details for your measurements.
Give formulas and references which provide a specific theoretical
context for your measurements.

\subsection{Our approach}

{\bf Brief historical context}
{\it (At most a paragraph, try to combine it with the next topic below)} 
As a context for your work,
you may want to relate what you are doing to first or previous 
work on this topic. Since you are doing an experimental work,
the context should be on the experimental technique. For example,
you may say that this was first done in a such and such way 
but later it was discovered
that one can also do it another way. Your technique may be
related to the first or none of the above.

\noindent
{\bf Your approach/technique}
{\it (At most a paragraph)}
What you have done, in broad terms. What's interesting and perhaps
different about it.

%=====================================================
%============ Section ==================================
%=====================================================

\section{Experimental setup} 

\subsection{Apparatus} 

Ideas behind the particular technique should be briefly
discussed. Enclose references. Sketches, pictures, and
suitable schematics should be included and explained
concisely. All major components of the system should be
mentioned and their role clearly described.


%=====================================================
%============ Importing pictures  ==========================
%=====================================================

% !! To be imported, all graphics must be converted !!
% !!    to encapsulated postscript (.eps format)    !!
% !!  The GNU Image Manipulation Program (GIMP) is  !!
% !!          capable of this conversion.           !!

\begin{figure}[h]
  %
  % placement specifier = { h,t,b,p,!,H }
  % see the following url for placement specifier definitions:
  % http://en.wikibooks.org/wiki/LaTeX/Floats,_Figures_and_Captions
  %
 \begin{center}
 \includegraphics*[2.75in,2.75in]{string_theory.eps}
  %
  %               [llx,lly][urx,ury]{ filename.eps }
  %
  % Note: llx = lower left x coordinate
  %       lly = lower left y coordinate
  %       urx = upper right x coordinate
  %       ury = upper right y coordinate
  %
  %       If [ llx,lly ] is omitted, [0,0] is assumed.
  %       Be sure to include units.
  %
 \caption{ My Caption, in all its glory.\label{fig:apparatus} }
 % See http://en.wikibooks.org/wiki/LaTeX/Labels_and_Cross-referencing
 %  for information on labels.
 \end{center}
\end{figure}

\subsection{Data Collection}

Data taking procedures should be described and various modes of
data collection explained. Calibration procedures and
relevant plots and numerical tables should be included.
State clearly what measurements were taken for the final
data analysis. Describe `doing the experiment' so it would
be helpful to other students in the future. This may need
to include physics arguments {\em what } and {\em how } data should
be collected.

%=====================================================
%============ Section ==================================
%=====================================================

\section{Data Analysis and Results}

A convenient means of thinking how to structure this section and what to 
include here is to decide which numbers, figures, or plots constitute the
essential output of this work. Then, describe logically elements of procedure(s)
which have led to obtain these ``products".


\subsection{Data Processing and Hypothesis Testing}

Describe data analysis. Details! Perhaps include a figure and refer
the reader to it.  Our apparatus, depicted in Figure~\ref{fig:apparatus},
was used to measure voltage from the probe. Maybe you will need
to include a table. Data, read out from the meter are included
in Table~\ref{tab:events}. They were used to calculate the velocity of
neutrinos using the formula:
$$ v_\nu = {L \over t},$$
where $L$ is the distance traveled by neutrinos in time $t$.

Describe calculations of the final results.
Thoroughly address error analysis and discussion of measurement
uncertainties. Remember: NO EXPERIMENTAL RESULT CAN BE QUOTED
WITHOUT AN ERROR (UNCERTAINTY)! 
USE CORRECT NUMBER OF SIGNIFICANT FIGURES!
Do not forget about random or systematic
uncertainties. Be sure to propagate errors correctly!
Include a demonstrative graph when possible.
%See Figure~\ref{fig:results}.
Discuss sources of random and systematic uncertainties.

Make final assessment and interpretation after that.
Discuss apparatus problems if any. Suggestions for
lab setup or approach improvements are welcome!

%=====================================================
%============ Importing pictures  ====================
%=====================================================

% !! To be imported, all graphics must be converted !!
% !!    to encapsulated postscript (.eps format)    !!
% !!  The GNU Image Manipulation Program (GIMP) is  !!
% !!          capable of this conversion.           !!

\begin{figure}[h]
  %
  % placement specifier = { h,t,b,p,!,H }
  % see the following url for placement specifier definitions:
  % http://en.wikibooks.org/wiki/LaTeX/Floats,_Figures_and_Captions
  %
 \begin{center}
 \includegraphics*[2.25in,3.2in]{centrifugal_force.eps}
  %
  %               [llx,lly][urx,ury]{ filename.eps }
  %
  % Note: llx = lower left x coordinate
  %       lly = lower left y coordinate
  %       urx = upper right x coordinate
  %       ury = upper right y coordinate
  %
  %       If [ llx,lly ] is omitted, [0,0] is assumed.
  %       Be sure to include units.
  %
 \caption{ The experiment yielded interesting results presented above. 
 The plot shows time as a function of distance traveled by all particles.
 Neutrinos are represented by solid triangles.~\label{fig:results} }
 % See http://en.wikibooks.org/wiki/LaTeX/Labels_and_Cross-referencing
 %  for information on labels.
 \end{center}
\end{figure}


%===========================================================================
%=========================== Table 1 =======================================
%===========================================================================
%
% Note: the position of the table does not always depend on its position here. See
% http://en.wikibooks.org/wiki/LaTeX/Tables
% for details.
%

\begin {table}[h]
{
{%\footnotesize
\begin {center}
\begin {tabular} {c | c c  c | c | c c }
\hline\hline
Run 			&   ~~POT~~ 		&
\multicolumn{2}{ | c } {Predicted}  &   \multicolumn{2}  {| c} {Selected} \\
Period		& $(10^{20})$	&
\multicolumn{2}{ | c } {(No oscillations)}  &   \multicolumn{2}  {| c} {(Far Detector)} \\
			    &
			& \multicolumn{1} {| c } {~~~Fully} & \multicolumn{1} { c } {~~~Partially} 
			& \multicolumn{1} {| c } {~~~Fully} & \multicolumn{1} { c } {~~~Partially} \\
			
\hline
I			& 1.269		
			& \multicolumn{1} {| r } {426 } & \multicolumn{1} { r } {375 } 
		     	& \multicolumn{1} {| r } {318 } & \multicolumn{1} { r } {357 } \\

II		     	& 1.943
			& \multicolumn{1} {| r } {639 } & \multicolumn{1} { r } {565 } 
		    	& \multicolumn{1} {| r } {511 } & \multicolumn{1} { r } {555 } \\

\hline
Total			& 7.246
			& \multicolumn{1} {| r } {2,451 } & \multicolumn{1} { r } {2,206 } 
		     	& \multicolumn{1} {| r } {1,986 } & \multicolumn{1} { r } {2,017 } \\

\hline% \hline
\end {tabular}
\end {center}
}
}
\caption {\label{tab:events}
Predicted and observed numbers of events classified in the Far Detector as fully and
partially reconstructed charged current interactions shown for all running periods.
 }
\end {table}


\subsection{Results and Brief Discussion}

Clearly present the result of your analysis. Make sure
you include the uncertainties. No experimental result
can be quoted without an error attached to it.

Your results should be compared (discussed) with predictions and other
measurements. The main results should be then used in the
abstract.

%=====================================================
%============ Section ==================================
%=====================================================

\section{Summary and conclusions}

Summarize briefly the results of the experiment. Our value 
of $(v_\nu - c)/c  = (0.05\pm0.12)$  is consistent
with measurements conducted by other experiments~\cite{OtherExpts}

{\bf Acknowledgements:} (i.e., thank for) contributions or help
of your partner(s) and or
others (TA, machine shop, software used, ...).

%=====================================================
%============ Bibliography  ==============================
%=====================================================

\begin{thebibliography}{9}

\bibitem{book} 
R.~Feynman, {\it QED}, Ch.7.

\bibitem{article}	
R.~Dalitz, Proc. Roy. Soc. (London) {\bf A64}, 667 (1951)

\bibitem{OtherExpts} 
P.~Adamson {\it et al.}  [MINOS Collaboration],
  ``Measurement of neutrino velocity with the MINOS detectors and NuMI neutrino beam,''
  Phys.\ Rev.\ D {\bf 76}, 072005 (2007)
  [arXiv:0706.0437 [hep-ex]].
  %%CITATION = ARXIV:0706.0437;%%

\end{thebibliography}

%=====================================================
%============ End ====================================
%=====================================================

\end{document}

%=====================================================
%============ End ====================================
%=====================================================
