\documentclass{article}
\usepackage[latin1]{inputenc}
\usepackage{enumerate}
\usepackage{hyperref}
\usepackage{graphics}
\usepackage{graphicx}
\usepackage{caption}
\usepackage{subcaption}
\usepackage{tabularx}
%\usepackage{amsmath}
% enumerate is numbered \begin{enumerate}[(I)] is cap roman in parens
% itemize is bulleted \begin{itemize}
% subfigures:
% \begin{subfigure}[b]{0.5\textwidth} \includegraphics{asdf.jpg} \caption{} \label{subfig:asdf} \end{subfigure}
\hypersetup{colorlinks=true, urlcolor=blue, linkcolor=blue, citecolor=red}
\graphicspath{ {C:/Users/Evan/Desktop/} }
\title{Assignment 1: Week 2}
\author{Evan Ott}
%\date{DATE}
\setcounter{secnumdepth}{0}
\usepackage[parfill]{parskip}
\begin{document}
\maketitle

\begin{section}{Math}
	In this week's session, we learned about using the math mode of \LaTeX (isn't that fancy? - you can type it by saying $\backslash\texttt{LaTeX}$). You can enter math mode for inline text with any of the following:
\begin{verbatim}
	\begin{math}...\end{math}
	\(...\)
	$...$
\end{verbatim}
And you can do out-of-line equations with:
\begin{verbatim}
	\begin{equation}...\end{equation}
	\[...\]
	$$...$$
\end{verbatim}

For your first exercise, please produce the following equations:

\Large
\begin{equation}
	\int_\alpha^\beta\frac{\partial^2\Psi}{\partial{x^2}}d\tau
\end{equation}
\begin{equation}
	\label{eq:snell}
	d\sin(\theta)=\pm{\sqrt[3]{n^3}}\lambda
\end{equation}
\begin{equation}
	2\Gamma_2+\Omega_2\rightarrow 2\Gamma_2\Omega
\end{equation}
\normalsize(HINT: use $\backslash\texttt{partial}$, $\backslash\texttt{pm}$, and $\backslash\texttt{rightarrow}$).

Note: for the equation for Snell's law (Equation \ref{eq:snell} $\leftarrow$ isn't that link cool? We'll do those next week), be sure you get $\sin(\theta)$ not $sin(\theta)$.

If you feel ambitious, check out one of the symbol sheets and try to write something a little different:

\Large
\begin{equation}
	\int_{\mathcal{S}}\nabla^2V
\end{equation}
\begin{equation}
	\sum_{i=1}^{n}(\int{f_i(x)dx})=\int(\sum_{i=1}^{n}f_i(x))dx
\end{equation}
\begin{equation}
	\widehat{H}\Psi=i\hbar\frac{\partial\Psi}{\partial{t}}=-\frac{\hbar^2}{2m}\nabla^2\Psi+V\Psi=E\Psi
\end{equation}
\end{section}

\begin{section}{Pictures}

For now, if you can get the image below with the instructions given, you're right on track.
\begin{enumerate}[1.]
	\item{In the preamblel to the document, include the packages $\texttt{graphics}$ and $\texttt{graphicx}$. $\texttt{graphics}$ is covered in the tutorial, and  $\texttt{graphicx}$ allows you to use other image formats in addition to $\texttt{*.ps}$.}
	\item{Download \url{http://www.ph.utexas.edu/~sps/pics/sps_logo.png} and put it in a folder with path $path$.}
	\item{In the preamble, tell \LaTeX where the image you just downladed is with $\backslash{\texttt{graphicspath}\{ \{path\} \}}$. 
	For example, you might have (on Windows):	\\
	$\backslash{\texttt{graphicspath\{ \{C:/Users/Evan/Downloads/\} \}}}$.}
	\item{Now, include it in your document with the following:}
\end{enumerate}
\begin{verbatim}
\begin{figure}[h]
    \includegraphics{sps_logo.png}
\end{figure}
\end{verbatim}
and you should end up with:

\begin{figure}[h]
	\includegraphics{sps_logo.png}
\end{figure}

The $\texttt{[h]}$ in the code above simply tells \LaTeX to put the image at this point in the document - generally, it may move it to other pages if you don't tell it what to do 
(which is often the right decision aesthetically speaking).
\end{section}

\begin{section}{Tables and Matrices}

I'm going to keep this section short because it's kinda tedious. Try to write the Pauli spin matrices in \LaTeX:

\Large
\begin{equation}
	\overline{S}=\left(\begin{array}{ccc}
		S_x & S_y & S_z \end{array} \right)
\end{equation}
\begin{equation}
		S_x=\frac{\hbar}{2} \left( \begin{array}{cc}
		0 & 1 \\
		1 & 0 \end{array}\right)
\end{equation}
\begin{equation}
	S_y=\frac{\hbar}{2} \left( \begin{array}{cc}
		0 & -i \\
		i & 0 \end{array} \right)
\end{equation}
\begin{equation}
	S_z=\frac{\hbar}{2}\left(\begin{array}{cc}
		1 & 0 \\
		0 & -1 \end{array}\right)
\end{equation}

\normalsize
For this next part, you will also need the $\texttt{tabularx}$ package included in the preamble (it gives you the horizontal lines and multiple column entries).

Now, try to write this simple data set (mind the justification!):

\Huge

\begin{tabular}{ l | c || r}
\multicolumn{3}{c}{Fibonacci Numbers}\\
\hline 
x & y & x+y \\ 
\hline
1 & 1 & 2 \\
1 & 2 & 3 \\
2 & 3 & 5 \\
3 & 5 & 8 \\
... & ... & ... \\
\hline
17711 & 28657 & 46368 
%\hline
\end{tabular}
\end{section}
\normalsize

\begin{thebibliography}{9}
\bibitem{sps}
	SPS \LaTeX Home Page \\
	\url{http://www.ph.utexas.edu/~sps/LaTeX/}
\bibitem{symbols}
	Recommended symbol sheet \\
	\url{http://amath.colorado.edu/documentation/LaTeX/Symbols.pdf}
\bibitem{more}
	Wikibooks on \LaTeX \\
	\url{en.wikibooks.org/wiki/LaTeX}
\end{thebibliography}
\end{document}