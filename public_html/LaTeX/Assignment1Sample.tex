\documentclass[onecolumn]{article}
\usepackage{amsmath}
\usepackage{hyperref}
\usepackage{enumerate}
\usepackage{graphics}
\usepackage{graphicx}
\usepackage{tabularx}
\graphicspath{ {C:/Users/Evan/Desktop/} }
\hypersetup{colorlinks=true, urlcolor=blue, linkcolor=blue, citecolor=red}
\title{Assignment 1 (Or something like it)}
\author{Evan Ott}
\date{\today}
\begin{document}
\maketitle
\begin{abstract}
	The purpose of this document is to provide examples somewhat similar as to what is seen in the assignment for if you're having difficulty with it. The ``answers'' for some of the 
	real problems on the assignment are going
	to be easier than what is represented here, because this document contains some information we haven't covered yet. I encourage you to take a look if you're interested 
	though. You might even learn some physics!
\end{abstract}
\begin{section}{Math Formatting}
	\begin{subsection}{Easy problems}
		\begin{equation}
			\int_{0}^{\pi}\frac{\partial{\Omega}}{\partial{t}}d\kappa
		\end{equation}
		\begin{equation}
			\frac{d\tan(\theta)}{d\theta}=\sec^2{\theta}
		\end{equation}
		\begin{equation}
			n=\pm\sqrt[2]{n^2}\rightarrow n=0
		\end{equation}
	\end{subsection}
	\begin{subsection}{Harder problems}
		\begin{equation}
			V=-\nabla{E}
		\end{equation}
		\begin{equation}
			f(x)=\sum_{i=0}^{\infty}\frac{f^{(i)}(x-a)\times(x-a)^i}{i!}
		\end{equation}
		\begin{equation}
			\widehat{p}\Psi_p=-i\hbar\frac{\partial\Psi_p}{\partial{x}}=p\Psi_p
		\end{equation}
	\end{subsection}
\end{section}

\begin{section}{Images}
	\begin{figure}[h]
		\includegraphics[width=.25in, height=1in]{sps_logo.png}
	\end{figure}
	In \LaTeX~if you put a $\backslash\backslash$ somewhere, then it'll start a new line. Otherwise, if your text in the code is
	on successive lines, it'll assume you meant to keep things in one paragraph. If you skip two lines,

	it'll start the next paragraph.
	\begin{figure}[h]
		\includegraphics[width=1in, height=.25in]{sps_logo.png}
	\end{figure}
	We'll do more with images later. For now, you might want to Google ``LaTeX includegraphics options'' for more info.
\end{section}

\begin{section}{Tables and Matrices}
	\begin{subsection}{Matrices}
		This code gets a little technical, but if you format it nicely, it's totally manageable. In my code for this section, I put each row of the matrix on a separate line
		of code so that changing it later would be easier.

		In an effort to teach some physics, I'll use the simple Hooke's Law spring for my matrix example.

		For the simple spring, we know that $\sum{\vec{F}}=-kx=m\vec{a}$. Well, we can also express that as:
		$$-kx=m\ddot{x}$$ where \(\ddot{x}=\frac{d^2x}{dt^2}\). Using this notation, we have that (using something we haven't
		talked about for formatting, but we'll do that soon - look at the $\texttt{amsmath}$ package, specifically the $\texttt{align}$ environment):
		\begin{align}
			\frac{dx}{dt}&=\dot{x}	\\
			\frac{d(m\dot{x})}{dt}&=-kx
		\end{align}
		
		We can then express these equations together as:
		\begin{equation}
			\frac{d}{dt}
			\left(\begin{array}{c}
				x \\
				\dot{x} 
			\end{array}\right) = 
			\left(\begin{array}{cc}
				0 & 1 \\
				-\frac{k}{m} & 0
			\end{array}\right)\left(\begin{array}{c}
				x \\
				\dot{x}
			\end{array}\right)
		\end{equation}
		To solve this set of equations, we can find the eigenvalues from
		\begin{equation}
			\det\left(\left(\begin{array}{cc}
				0 & 1 \\
				-\frac{k}{m} & 0
			\end{array}\right) -\lambda\left(\begin{array}{cc}
				1 & 0 \\
				0 & 1
			\end{array}\right)\right)=\left|\begin{array}{cc}
				-\lambda & 1 \\
				-\frac{k}{m} & -\lambda
			\end{array}\right|=0
		\end{equation}
		Which gives
		$$\lambda^2=-\frac{k}{m}$$ or the known normal mode of the simple harmonic oscillator,
		\[\lambda=i\omega_0=\pm{i}\sqrt{\frac{k}{m}}\]

		We can then use this result to determine the general solution for $x$:
		\begin{align}
			\left(\begin{array}{cc}
				0 & 1 \\
				-\frac{k}{m} & 0
			\end{array}\right)&\left(\begin{array}{c}
				x_1 \\
				\dot{x_1}
			\end{array}\right) = i\omega_0\left(\begin{array}{c}
				x_1 \\
				\dot{x_1}
			\end{array}\right) \\
			\dot{x_1}&=\frac{dx_1}{dt}=i\omega_0{x_1}	\\
			\int\frac{dx_1}{x_1}&=\int{i\omega_0}dt \\
			\ln{x_1}&=i\omega_0{t}+c_1 \\
			x_1(t)&=Ce^{i\omega_0{t}}
		\end{align}
		The solution for \begin{math}\lambda=-i\omega_0\end{math} is almost identical, yielding
		\begin{equation}
			x_2(t)=Be^{-i\omega_0{t}}
		\end{equation}
		So the general solution $x(t)=x_1(t)+x_2(t)$ can be obtained by employing Euler's identity $$e^{ix}=\cos(x)+i\sin(x)$$ and condidering that our constants C and B above are arbitrary and may be complex:
		\begin{align}
			\label{eq:alignment}
			x(t)&=Ce^{i\omega_0{t}}+Be^{-i\omega_0{t}} \\
			&=(C+B)\cos(\omega_0{t})+i(C-B)\sin(\omega_0{t})\\
			&=A_1\cos(\omega_0{t})+A_2\sin(\omega_0{t})
		\end{align}
		Thus, we have arrived at the general solution to the undamped, unforced simple spring in one dimension. The two cool parts are that:
		\begin{enumerate}[a)]
			\item{We took what is normally a single second-order linear differential equation and turned it into a system of two first-order linear differential equations, then solved
				them using the techniques of linear algebra.}
			\item{We just used all kinds of matrices and other math features in a \LaTeX~document and made the derivation beautiful!}
		\end{enumerate}
	\end{subsection}
	
	\begin{subsection}{Tables}
		Because that last section was {\Huge{Gigantic}}, I'm going to keep this one shorter.

		\large
		\begin{tabular}{l || c | c | r}
			\multicolumn{4}{c}{Sums of powers}	\\
			\hline 
			k & $\sum_{i=1}^{k}(i^0)$ & $\sum_{i=1}^{k}(i^1)$ & $\sum_{i=1}^{k}(i^2)$ \\
			\hline
			1 & 1 & 1 & 1 \\
			2 & 2 & 3 & 5 \\
			3 & 3 & 6 & 14 \\
			4 & 4 & 10 & 30 \\
			5 & 5 & 15 & 55 \\
			... & ... & ... & ... \\
			\hline
			$n$ & $n$ & $\frac{n(n+1)}{2}$ & $\frac{n(n+1)(2n+1)}{6}$
		\end{tabular}
		\\ \\
		One thing to note with tables is that they are in plain text mode, but you can use math with the normal $\texttt{\$...\$}$ convention.
	\end{subsection}
\end{section}

\begin{section}{Conclusion}
	This section shows some of the power of \LaTeX~to do math-intensive things beautifully, with relatively low effort (once you get used to things). Some notes on things I've done in
	this document that we haven't covered follow.
	\begin{subsection}{Code within code}
		If you're wondering how I get the cool font that I associate with things you will write in code, there are two ways to do this. The one for inline code is to use the command
		$\backslash{\texttt{texttt\{...\}}}$ when in math mode or to use the code $\backslash{\texttt{verb+...+}}$ where `\verb_+_' is an arbitrary character on both ends
		of your code, which is not to be done in math mode. To do this for multiple lines, you can use the $\texttt{verbatim}$ environment as seen below:\\ \\
		\verb+\begin{verbatim}+ \\
		\verb+    code here...+ \\
		\verb+\end{verbatim}+
		\\ \\
		(The careful reader of my code will note that for the above lines, I used \verb+\verb+ and not \verb+\begin{verbatim}+ for this because putting ``\verb+\end{verbatim}+''
		within a \verb+verbatim+ is impossible.)
	\end{subsection}
	\begin{subsection}{Extra packages}
		As mentioned in the assignment, the \verb+graphicx+ and \verb+tabularx+ packages are great for making graphics and tables more flexible. In this document, I used
		the \verb+amsmath+ package for the \verb+align+ environment, which lets you put multiple equations in a single section, aligning them vertically based on specific
		places that you specify. I also used the \verb+hyperref+ package to do some labelling for equations.
		\begin{subsubsection}{The $\texttt{align}$ Environment}
			The \verb+align+ environment is easy to use and can come in handy if you're presenting a derivation or simplification of some sort. I used this most representatively
			in Equation \ref{eq:alignment}, where I aligned the equals signs for the three equations. The following code:
			\begin{verbatim}
			\begin{align}
			    x&=1+2+3+4+5+6+y \\
			    y=a+b+c+d+e+f&+g
			\end{align}
			\end{verbatim}
			uses the `$\texttt{\&}$' symbol to mark where the equations should be aligned, as seen below:
			\begin{align}
			    x&=1+2+3+4+5+6+y \\
			    y=a+b+c+d+e+f&+g
			\end{align}
		\end{subsubsection}
		\begin{subsubsection}{The $\texttt{hyperref}$ Package}
			We'll talk more about this package later, but for now, what I've done is use labels within specific environments, then reference them later. The preamble to this
			document contains an extra line with $\backslash\texttt{hypersetup\{...\}}$ that sets the link colors how I prefer them. In use, I can create a label pretty
			much anywhere, which will look something like \verb+\label{environment_type:label_name}+ then I reference it with \verb+\ref{environment_type:label_name}+.
			For Equation \ref{eq:alignment} that I've referenced before, I used \verb+\label{eq:alignment}+ and \verb+\ref{eq:alignment}+ to get the job done.
		\end{subsubsection}
	\end{subsection}
	\begin{subsection}{Final Thoughts}
		At this point, you may note that my document has more than 200 lines of code, which is a lot. But you should note that this has produced 5 pages of information,
		with many lines being either just text or simple parts of equations. I hope this helps demonstrate the power of \LaTeX, and hope you'll join us for our \LaTeX seminar
		if you can (check our website at \url{www.ph.utexas.edu/~sps/} for date/time info). If you have questions, feel free to email the officers about \LaTeX~at 
		\href{mailto:spsofficers@gmail.com}{spsofficers@gmail.com} $\leftarrow$ opens in the default mail application, the address is also just spsofficers@gmail.com .\\ \\
		Don't Forget To Be Awesome!
	\end{subsection}
\end{section}
\end{document}